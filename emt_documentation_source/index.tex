% Options for packages loaded elsewhere
\PassOptionsToPackage{unicode}{hyperref}
\PassOptionsToPackage{hyphens}{url}
\PassOptionsToPackage{dvipsnames,svgnames,x11names}{xcolor}
%
\documentclass[
  letterpaper,
  DIV=11,
  numbers=noendperiod]{scrreprt}

\usepackage{amsmath,amssymb}
\usepackage{iftex}
\ifPDFTeX
  \usepackage[T1]{fontenc}
  \usepackage[utf8]{inputenc}
  \usepackage{textcomp} % provide euro and other symbols
\else % if luatex or xetex
  \usepackage{unicode-math}
  \defaultfontfeatures{Scale=MatchLowercase}
  \defaultfontfeatures[\rmfamily]{Ligatures=TeX,Scale=1}
\fi
\usepackage{lmodern}
\ifPDFTeX\else  
    % xetex/luatex font selection
\fi
% Use upquote if available, for straight quotes in verbatim environments
\IfFileExists{upquote.sty}{\usepackage{upquote}}{}
\IfFileExists{microtype.sty}{% use microtype if available
  \usepackage[]{microtype}
  \UseMicrotypeSet[protrusion]{basicmath} % disable protrusion for tt fonts
}{}
\makeatletter
\@ifundefined{KOMAClassName}{% if non-KOMA class
  \IfFileExists{parskip.sty}{%
    \usepackage{parskip}
  }{% else
    \setlength{\parindent}{0pt}
    \setlength{\parskip}{6pt plus 2pt minus 1pt}}
}{% if KOMA class
  \KOMAoptions{parskip=half}}
\makeatother
\usepackage{xcolor}
\setlength{\emergencystretch}{3em} % prevent overfull lines
\setcounter{secnumdepth}{5}
% Make \paragraph and \subparagraph free-standing
\ifx\paragraph\undefined\else
  \let\oldparagraph\paragraph
  \renewcommand{\paragraph}[1]{\oldparagraph{#1}\mbox{}}
\fi
\ifx\subparagraph\undefined\else
  \let\oldsubparagraph\subparagraph
  \renewcommand{\subparagraph}[1]{\oldsubparagraph{#1}\mbox{}}
\fi


\providecommand{\tightlist}{%
  \setlength{\itemsep}{0pt}\setlength{\parskip}{0pt}}\usepackage{longtable,booktabs,array}
\usepackage{calc} % for calculating minipage widths
% Correct order of tables after \paragraph or \subparagraph
\usepackage{etoolbox}
\makeatletter
\patchcmd\longtable{\par}{\if@noskipsec\mbox{}\fi\par}{}{}
\makeatother
% Allow footnotes in longtable head/foot
\IfFileExists{footnotehyper.sty}{\usepackage{footnotehyper}}{\usepackage{footnote}}
\makesavenoteenv{longtable}
\usepackage{graphicx}
\makeatletter
\def\maxwidth{\ifdim\Gin@nat@width>\linewidth\linewidth\else\Gin@nat@width\fi}
\def\maxheight{\ifdim\Gin@nat@height>\textheight\textheight\else\Gin@nat@height\fi}
\makeatother
% Scale images if necessary, so that they will not overflow the page
% margins by default, and it is still possible to overwrite the defaults
% using explicit options in \includegraphics[width, height, ...]{}
\setkeys{Gin}{width=\maxwidth,height=\maxheight,keepaspectratio}
% Set default figure placement to htbp
\makeatletter
\def\fps@figure{htbp}
\makeatother
% definitions for citeproc citations
\NewDocumentCommand\citeproctext{}{}
\NewDocumentCommand\citeproc{mm}{%
  \begingroup\def\citeproctext{#2}\cite{#1}\endgroup}
\makeatletter
 % allow citations to break across lines
 \let\@cite@ofmt\@firstofone
 % avoid brackets around text for \cite:
 \def\@biblabel#1{}
 \def\@cite#1#2{{#1\if@tempswa , #2\fi}}
\makeatother
\newlength{\cslhangindent}
\setlength{\cslhangindent}{1.5em}
\newlength{\csllabelwidth}
\setlength{\csllabelwidth}{3em}
\newenvironment{CSLReferences}[2] % #1 hanging-indent, #2 entry-spacing
 {\begin{list}{}{%
  \setlength{\itemindent}{0pt}
  \setlength{\leftmargin}{0pt}
  \setlength{\parsep}{0pt}
  % turn on hanging indent if param 1 is 1
  \ifodd #1
   \setlength{\leftmargin}{\cslhangindent}
   \setlength{\itemindent}{-1\cslhangindent}
  \fi
  % set entry spacing
  \setlength{\itemsep}{#2\baselineskip}}}
 {\end{list}}
\usepackage{calc}
\newcommand{\CSLBlock}[1]{\hfill\break\parbox[t]{\linewidth}{\strut\ignorespaces#1\strut}}
\newcommand{\CSLLeftMargin}[1]{\parbox[t]{\csllabelwidth}{\strut#1\strut}}
\newcommand{\CSLRightInline}[1]{\parbox[t]{\linewidth - \csllabelwidth}{\strut#1\strut}}
\newcommand{\CSLIndent}[1]{\hspace{\cslhangindent}#1}

\KOMAoption{captions}{tableheading}
\makeatletter
\@ifpackageloaded{bookmark}{}{\usepackage{bookmark}}
\makeatother
\makeatletter
\@ifpackageloaded{caption}{}{\usepackage{caption}}
\AtBeginDocument{%
\ifdefined\contentsname
  \renewcommand*\contentsname{Table of contents}
\else
  \newcommand\contentsname{Table of contents}
\fi
\ifdefined\listfigurename
  \renewcommand*\listfigurename{List of Figures}
\else
  \newcommand\listfigurename{List of Figures}
\fi
\ifdefined\listtablename
  \renewcommand*\listtablename{List of Tables}
\else
  \newcommand\listtablename{List of Tables}
\fi
\ifdefined\figurename
  \renewcommand*\figurename{Figure}
\else
  \newcommand\figurename{Figure}
\fi
\ifdefined\tablename
  \renewcommand*\tablename{Table}
\else
  \newcommand\tablename{Table}
\fi
}
\@ifpackageloaded{float}{}{\usepackage{float}}
\floatstyle{ruled}
\@ifundefined{c@chapter}{\newfloat{codelisting}{h}{lop}}{\newfloat{codelisting}{h}{lop}[chapter]}
\floatname{codelisting}{Listing}
\newcommand*\listoflistings{\listof{codelisting}{List of Listings}}
\makeatother
\makeatletter
\makeatother
\makeatletter
\@ifpackageloaded{caption}{}{\usepackage{caption}}
\@ifpackageloaded{subcaption}{}{\usepackage{subcaption}}
\makeatother
\ifLuaTeX
  \usepackage{selnolig}  % disable illegal ligatures
\fi
\usepackage{bookmark}

\IfFileExists{xurl.sty}{\usepackage{xurl}}{} % add URL line breaks if available
\urlstyle{same} % disable monospaced font for URLs
\hypersetup{
  pdftitle={The Experimental Music Toolbox (EMT)},
  pdfauthor={M. Edward (Ed) Borasky},
  colorlinks=true,
  linkcolor={blue},
  filecolor={Maroon},
  citecolor={Blue},
  urlcolor={Blue},
  pdfcreator={LaTeX via pandoc}}

\title{The Experimental Music Toolbox (EMT)}
\author{M. Edward (Ed) Borasky}
\date{2024-05-14}

\begin{document}
\maketitle

\renewcommand*\contentsname{Table of contents}
{
\hypersetup{linkcolor=}
\setcounter{tocdepth}{2}
\tableofcontents
}
\bookmarksetup{startatroot}

\chapter*{Introduction}\label{introduction}
\addcontentsline{toc}{chapter}{Introduction}

\markboth{Introduction}{Introduction}

Welcome to the Experimental Music Toolbox! As the name suggests, the
Experimental Music Toolbox (EMT) is a collection of tools for creating
experimental music. But since it's based on open source Linux tools, it
can help you make any kind of music.

\section*{Who is EMT for?}\label{who-is-emt-for}
\addcontentsline{toc}{section}{Who is EMT for?}

\markright{Who is EMT for?}

The main audience for EMT is experienced coders who want to learn how to
make music using a computer. However, its core is the Ubuntu 22.04 LTS
Linux distribution, which means you can install any software from that
distribution, as well as any software provided by third parties. If you
wish, you can install a spreadsheet and do your taxes on it.

\section*{What's in it?}\label{whats-in-it}
\addcontentsline{toc}{section}{What's in it?}

\markright{What's in it?}

EMT is a Git repository housing a collection of scripts. The scripts can
install the following tools:

\begin{itemize}
\item
  Linux music packages: the main packages in this group are classic,
  low-level music programming environments like Csound, Pure Data and
  Supercollider.
\item
  High-level programming language integrated development environments
  (IDEs):

  \begin{itemize}
  \tightlist
  \item
    RStudio Server, with optional package development and audio analysis
    / synthesis tools. This is my primary tool set.
  \item
    JupyterLab, with PyTorch, torchaudio and the complete Python data
    science stack.
  \end{itemize}
\item
  The ChucK live coding audio programming language, built from source:
  Although ChucK is included in the Ubuntu 22.04 LTS repositories and
  could be installed with the Linux music packages, the version there is
  quite old. ChucK received a significant upgrade to version 1.5 since
  Ubuntu 22.04 was released, so scripts to build ChucK from source are
  included.
\item
  The NVIDIA CUDA Toolkit: If your system has an NVIDIA GPU, you can
  install the CUDA toolkit and program the GPU in C / C++.
\end{itemize}

\section*{What hardware do I need?}\label{what-hardware-do-i-need}
\addcontentsline{toc}{section}{What hardware do I need?}

\markright{What hardware do I need?}

EMT should run on any modern 64-bit PC (\texttt{amd64} /
\texttt{X86\_64}) capable of supporting Windows 11. There are three ways
to run it:

\begin{enumerate}
\def\labelenumi{\arabic{enumi}.}
\tightlist
\item
  On a Windows 11 machine using a Windows Subsystem for Linux
  installation of Ubuntu 22.04 LTS ``Jammy Jellyfish''.
\item
  As a Distrobox container inside any \texttt{amd64} / \texttt{X86\_64}
  Linux system running Distrobox 1.7.1.0 or later.
\item
  On a Windows 11-capable machine running Ubuntu 22.04. Note that I do
  not own such a machine so I can't verify that it works.
\end{enumerate}

The JupyterLab / PyTorch subsystem can be installed in either of two
modes: CPU and CUDA. The CPU mode will work on any \texttt{amd64} /
\texttt{X86\_64} system, but the CUDA mode requires an NVIDIA 10 series
GPU or newer. I test on a GTX 1650 and an RTX 3090. And the CUDA toolkit
only works with an NVIDIA GPU.

\section*{Plan of the book}\label{plan-of-the-book}
\addcontentsline{toc}{section}{Plan of the book}

\markright{Plan of the book}

I'll start with the easiest way to get started: a Windows 11PC with no
GPU. That involves just installing Ubuntu 22.04 from the Microsoft
Store. Part 1 will cover all the CPU mode tools in that environment.

Part 2 will cover the tools that use an NVIDIA CPU: the CUDA toolkit and
the GPU-enabled version of PyTorch. And Part 3 will cover Distrobox and
Fedora Atomic Desktops.

\part{Part 1: Windows Subsystem for Linux CPU Mode}

\chapter{Getting Started on Windows 11 WSL Ubuntu
22.04}\label{getting-started-on-windows-11-wsl-ubuntu-22.04}

\chapter{Linux Music Packages}\label{linux-music-packages}

\chapter{RStudio Server}\label{rstudio-server}

\chapter{JupyterLab}\label{jupyterlab}

\chapter{PyTorch and Torchaudio}\label{pytorch-and-torchaudio}

\chapter{ChucK from Source}\label{chuck-from-source}

\part{Part 2: NVIDIA GPUs in CUDA Mode}

\chapter{GPU-enabled PyTorch}\label{gpu-enabled-pytorch}

\chapter{CUDA Toolkit}\label{cuda-toolkit}

\part{Part 3: Distrobox Containers / Fedora Atomic Desktops}

\chapter{Pet Containers and
Distrobox}\label{pet-containers-and-distrobox}

\chapter{Fedora Atomic Desktops}\label{fedora-atomic-desktops}

\bookmarksetup{startatroot}

\chapter*{Summary}\label{summary}
\addcontentsline{toc}{chapter}{Summary}

\markboth{Summary}{Summary}

\bookmarksetup{startatroot}

\chapter*{References}\label{references}
\addcontentsline{toc}{chapter}{References}

\markboth{References}{References}

\phantomsection\label{refs}
\begin{CSLReferences}{0}{1}
\end{CSLReferences}



\end{document}
